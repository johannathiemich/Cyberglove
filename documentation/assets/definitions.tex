% Some makros used from https://github.com/Jintzo

\numberwithin{equation}{subsection}

% -- MATH MACROS
\newcommand{\thistheoremname}{}
\newcommand{\R}{\mathbb{R}}
\newcommand{\N}{\mathbb{N}}
\newcommand{\Q}{\mathbb{Q}}
\newcommand{\C}{\mathbb{C}}
\newcommand{\Z}{\mathbb{Z}}
\newcommand{\F}{\mathbb{F}}
\newcommand{\mi}{\mathrm{i}}

% -- for math in headings
\addtokomafont{disposition}{\boldmath}

% -- THEOREMS
\swapnumbers
\theoremstyle{definition}
\newtheorem*{df}{Definition}
\newtheorem*{sz}{Satz}
\newtheorem*{hs}{Hilfssatz}
\newtheorem*{kr}{Korrolar}
\newtheorem*{lm}{Lemma}
\newtheorem*{fl}{Folgerung}
\newtheorem*{bm}{Bemerkung}
\newtheorem*{genericthm}{\thistheoremname}
\newtheorem*{bsp}{Beispiel}
\newenvironment{bw}[1][\textbf{Beweis}]{\begin{proof}[#1]}{\end{proof}}

% -- task list
\mdfdefinestyle{mdtaskstyle}{%
	frametitlerule=true,
	outerlinewidth=2pt,
	roundcorner=0pt,
	innerrightmargin=0pt,
	innerleftmargin=0pt,
	innertopmargin=0pt,
	innerbottommargin=0pt,
	backgroundcolor=White,
	subtitleaboveskip=0pt,
	subtitlebelowskip=5pt}

\newcommand{\taskcontent}[7]
{
	\def\arraystretch{1.5}
	\begin{tabularx}{\textwidth}{r | l l}
		& #1. #4 \\
		\hline\hline
		Time in hours: & 
		\parbox[t]{10.7cm}{
			\begin{tabularx}{\textwidth}{l | l}
				\parbox[t]{5cm}{Estimated:\\#2\vspace{.2cm}}
				&
				\parbox[t]{5cm}{Actually:\\#3\vspace{.2cm}}
			\end{tabularx}
		}
		\\
		\hline
		Task description: & \parbox[t]{10.7cm}{#5\vspace{.1cm}} \\
		\hline
		Issues: & \parbox[t]{10.7cm}{
			\ifthenelse{\equal{\detokenize{}}{\detokenize{#6}}}{None.}{#6}
		\vspace{.1cm}} \\
		\hline
		Result: & \parbox[t]{10.7cm}{
			\ifthenelse{\equal{\detokenize{}}{\detokenize{#7}}}{TODO NO RESULT DEFINED!!}{#7}
			\vspace{.1cm}} \\
	\end{tabularx}
}
% \task{Name of the task}
%	{hardware/software/usecase/coordination}
%	{People responsible for the task}
%	{Planned time in hours}
% 	{Is time in hours}
% 	{Planned range of date for the task}
% 	{Description of the task}
%	{Issues during task processing and solution approach}
%	{Result}
\newcommand{\task}[9]%
{%
	\ifthenelse{\equal{\detokenize{hardware}}{\detokenize{#2}}}{
		\begin{mdframed}[style=mdtaskstyle,linecolor=RoyalBlue,%
			subtitlebackgroundcolor=RoyalBlue,frametitlebackgroundcolor=RoyalBlue,
			frametitle={\color{White}\,\,\,Hardware: #1}]%
			\taskcontent{#3}{#4}{#5}{#6}{#7}{#8}{#9}%
		\end{mdframed}%
	}{}
	\ifthenelse{\equal{\detokenize{software}}{\detokenize{#2}}}{
		\begin{mdframed}[style=mdtaskstyle,linecolor=Mahogany,%
			subtitlebackgroundcolor=Mahogany,frametitlebackgroundcolor=Mahogany,
			frametitle={\color{White}\,\,\,Software: #1}]%
			\taskcontent{#3}{#4}{#5}{#6}{#7}{#8}{#9}%
		\end{mdframed}%
	}{}
	\ifthenelse{\equal{\detokenize{usecase}}{\detokenize{#2}}}{
		\begin{mdframed}[style=mdtaskstyle,linecolor=OliveGreen,%
			subtitlebackgroundcolor=OliveGreen,frametitlebackgroundcolor=OliveGreen,
			frametitle={\color{White}\,\,\,Use-Case: #1}]%
			\taskcontent{#3}{#4}{#5}{#6}{#7}{#8}{#9}%
		\end{mdframed}%
	}{}
	\ifthenelse{\equal{\detokenize{coordination}}{\detokenize{#2}}}{
		\begin{mdframed}[style=mdtaskstyle,linecolor=RoyalPurple,%
			subtitlebackgroundcolor=RoyalPurple,frametitlebackgroundcolor=RoyalPurple,
			frametitle={\color{White}\,\,\,Coordination: #1}]%
			\taskcontent{#3}{#4}{#5}{#6}{#7}{#8}{#9}%
		\end{mdframed}%
	}{}
	\ifthenelse{\equal{\detokenize{general}}{\detokenize{#2}}}{
	\begin{mdframed}[style=mdtaskstyle,linecolor=Orange,%
		subtitlebackgroundcolor=Orange,frametitlebackgroundcolor=Orange,
		frametitle={\color{White}\,\,\,General: #1}]%
		\taskcontent{#3}{#4}{#5}{#6}{#7}{#8}{#9}%
	\end{mdframed}%
	}{}
}


% Listings
\definecolor{mygreen}{rgb}{0,0.6,0}
\definecolor{mygray}{rgb}{0.5,0.5,0.5}
\definecolor{mymauve}{rgb}{0.58,0,0.82}
\lstset{ %
	backgroundcolor=\color{white},   % choose the background color; you must add \usepackage{color} or \usepackage{xcolor}; should come as last argument
	basicstyle=\small,        % the size of the fonts that are used for the code
	breakatwhitespace=false,         % sets if automatic breaks should only happen at whitespace
	breaklines=true,                 % sets automatic line breaking
	captionpos=b,                    % sets the caption-position to bottom
	commentstyle=\color{mygreen},    % comment style
	deletekeywords={...},            % if you want to delete keywords from the given language
	escapeinside={\%*}{*)},          % if you want to add LaTeX within your code
	extendedchars=true,              % lets you use non-ASCII characters; for 8-bits encodings only, does not work with UTF-8
	frame=single,	                   % adds a frame around the code
	keepspaces=true,                 % keeps spaces in text, useful for keeping indentation of code (possibly needs columns=flexible)
	keywordstyle=\color{blue},       % keyword style
	morekeywords={*,...},            % if you want to add more keywords to the set
	numbers=left,                    % where to put the line-numbers; possible values are (none, left, right)
	numbersep=5pt,                   % how far the line-numbers are from the code
	numberstyle=\tiny\color{mygray}, % the style that is used for the line-numbers
	rulecolor=\color{black},         % if not set, the frame-color may be changed on line-breaks within not-black text (e.g. comments (green here))
	showspaces=false,                % show spaces everywhere adding particular underscores; it overrides 'showstringspaces'
	showstringspaces=false,          % underline spaces within strings only
	showtabs=false,                  % show tabs within strings adding particular underscores
	stepnumber=1,                    % the step between two line-numbers. If it's 1, each line will be numbered
	stringstyle=\color{mymauve},     % string literal style
	tabsize=1,	                   % sets default tabsize to 2 spaces
	title=\lstname                   % show the filename of files included with \lstinputlisting; also try caption instead of title
}