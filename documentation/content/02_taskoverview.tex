\section{Task Overview}

\subsection{Our Approach}

The different project subgroups are represented by the tables listing their tasks, the time planned and needed to actually fulfill each task, task description, difficulties which came up while performing the tasks and the final result of each task.

General efforts of the VR internship for example attendance during the lectures and preparatory work, like VR homework is shown below the group tables. The hours are cumulated hours over all group members.

\subsection{Tasks}

% Define new tasks with the \task command:
% \task{Name of the task}
%	{keyword}		
%		%where keyword is one of the following: software,hardware,usecase,coordination
%	{People responsible for the task}
%	{Planned time in hours}
% 	{Is time in hours}
% 	{Planned range of date for the task}
% 	{Description of the task}
%	{Issues during task processing and solution approach}
%	{Result}

%hardware:
\subsubsection{Hardware tasks}
\task{Circuit board wiring}%%% Name of the task
	{hardware}%%% keyword
	{Fabian, Frieder}%%% People
	{15 hours each}%%% Planned time
	{21 hours each}%%% Taken time
	{09.11.17 - 30.11.17}%%% Range of date (Planned)
	{Getting information about the components of the board.}%%% Taskdescription
	{Information overload and learning the basic skills first.}%%% Issues, solution approach
	{Complete wiring for our circuit board.}%%% Result

\task{Design Board}%%% Name of the task
	{hardware}%%% keyword
	{Frieder}%%% People
	{20 hours}%%% Planned time
	{38 hours (5 hours first desing, 20 hours reducing size, 10 hours troubleshooting 3 hours setting labels)}%%% Taken time
	{30.11.17 - 07.12.17, 20.12.17}%%% Range of date (Planned)
	{With the schematics of the board, drawn in the step before, now a pcb layout as small as possible was needed. With manual placement and  Using of the autorouter, this was done in a few hours but to reduce the size of the board to a size by which it can easily fit on the back of someones hand replacing and manually routing was required.}%%% Taskdescription
	{Too many connection coming from the pin header with too narrow space between them to connect. Soultion: Trial and Error and connecting one Side of the chip to VCC and the other side to GND.  Some airwires (not not connected parts which should be connected) near parts which could not be resolved. Solution: Replace faulty parts with the right ones.}%%% Issues, solution approach
	{PCB wich was ready to solder after ordering.}%%% Result
	
\task{Solder the board (two times)}%%% Name of the task
	{hardware}%%% keyword
	{Fabian}%%% People
	{0 hours}%%% Planned time
	{30 hours}%%% Taken time
	{24.01.2018, 25.01.2018, 31.01.2018 - 02.02.2018}%%% Range of date (Planned)
	{Soldering the board with its components}%%% Taskdescription
	{We had trouble getting the board to work, problem was a missing wire connection. Then we accidently destroyed the first board and had to make another one without the direct help of Michael.}%%% Issues, solution approach
	{Working board.}%%% Result
	
\task{Burn bootloader to second board}%%% Name of the task
	{hardware}%%% keyword
	{Fabian, Frieder}%%% People
	{0 hours}%%% Planned time
	{8 hours each}%%% Taken time
	{02.02.18}%%% Range of date (Planned)
	{To burn the bootloader to the chip we had to direcly solder cables to the board which we then connected to a programmer device. Also exessive testingng with a multimeter, that there are no bridges.}%%% Taskdescription
	{We did not have a programmer. Solution: we bought one. It was very difficult to solder the wires which were so near to each other, approach use a magnifying glass. We got errors while trying to burn the bootloader. We used Amtel Stuio instead of the Arduino GUI, used a programm called zardig to install the right drivers to the computer and reconnected some of the cables to the board.}%%% Issues, solution approach
	{The board now behaves like a Arduino Leoanrdo and can be programmed via USB.}%%% Result
	
\task{Select battery}%%% Name of the task
	{hardware}%%% keyword
	{Fabian, Frieder}%%% People
	{2 hours each}%%% Planned time
	{1 hour each}%%% Taken time
	{09.11.17 - 07.12.17}%%% Range of date (Planned)
	{Selct a battery which fits to the board, regarding size, voltage and capacity.}%%% Taskdescription
	{Flat battery would be bigger than board, and to be orderd which would also take time. Solution: Use a small round battery which Michael alredy has}%%% Issues, solution approach
	{Small round battery which can be attached to the side of the board.}%%% Result
	
\task{Design of case for hardware}%%% Name of the task
	{hardware}%%% keyword
	{Frieder}%%% People
	{5 hours}%%% Planned time
	{36 hours (5 hours case for board, 5 hours later discarded battery housing, 12 hours discussing desired/needed changes and implement them, 8 hours inspecting test print and adjusting model, 6 hours mend 2nd print)}%%% Taken time
	{14.12.17 - 18.01.18}%%% Range of date (Planned)
	{Make a case which houses the board, carries tracking, a battery and can be easily mounted to the glove.}%%% Taskdescription
	{The tracking used on the old glove was too big for our goal of minimizing the size of the glove. Also even though it seemed 3D printed, no 3D models were avaiable, another problem was that the tracking cannot be printed directly on the case because that would require too much support matertial. So as a solution a new tracking had to be moddeled, printed and attached to the rest of the case. Adding a Battery underneath the board would make the case around 4 to 5 cm tall. It would also hinder the already difficult mounting mechanism. To solve these issues, a smaller, round battery will be attached to the side of the case, and instead of screwing the case to the glove the case will be held in place by magnets attached to the case and metal strips attached to the glove. \\
	When evalutating the test print the board did not fit in the case and the tracking had severe printing error, so the position of many holes could not be correcly evaluated. Because the first real print crashed the printer, there was only time left for a second print. Approach: Tthe measures which could be evaluated were corrected and the others were  ment after the second succesful print.}%%% Issues, solution approach
	{A 3D printed case with attachable tracking and battery.}%%% Result
	
\task{Assemble Case}%%% Name of the task
	{hardware}%%% keyword
	{Frieder}%%% People
	{3 hours}%%% Planned time
	{14 hours (3 hours connecting battery case, 2 hours magnets, 3 hour tracking, 2 hours going to the hardware-store, 4 hours trying to fit a switch to the board)}%%% Taken time
	{28.01.18 - 05.02.18}%%% Range of date (Planned)
	{Solder Cables to Baterry case, screw battery case and tracking to the rest of the case. Also place and screw the board in the case, add tracking points to the tracking holders and glue magnets in their place.}%%% Taskdescription
	{Cables of battery did fit through their hole in the case but were to thick with shrinking tubes. Solution: enlarge the holes. \\
	The support materials from printing did not come out properly out the wells for the magnets, so they didn't stayed in place. Solution: Carefull remove glue and support material and reattach magnets.\\
	Could not allocate the right screws. Approach: Go to two different hardware stores, buy the rest of the internet.\\
	The first switch was way too high. Approchach: Second switch, it was lower but very difficult to fit on the board. Because the switch was soldered on the fist board and too would only have fit in the case if  shortened, the second board was not equipped with a switch at all and is turned off by removing the easy removable battery.}%%% Issues, solution approach
	{A fully assembled case.}%%% Result
	
\task{Connect sensors and joystick to a plug strip}%%% Name of the task
	{hardware}%%% keyword
	{Frieder}%%% People
	{2 hours}%%% Planned time
	{10 hours}%%% Taken time
	{28.01.18 - 03.01.18}%%% Range of date (Planned)
	{To connect the bending, fingertipsensors and joystick easily to the board, they all have to be selodered to a plug which then can be connected with the pin socket of the board.}%%% Taskdescription
	{We used a very thin wire wich was, before connecting, sewed to the glove. So it was difficult to connect on the one side to the sensors and on the other side to the plug also the wire did break very easily. Solution approach: Use longer shrinking tubes and be very carefull.}%%% Issues, solution approach
	{Glove with attached sensors and one plug to connect them all to the case.}%%% Result
	
\task{New Glove Design}%%% Name of the task
	{hardware}%%% keyword
	{Tanja}%%% People
	{5 hours}%%% Planned time
	{6 hours}%%% Taken time
	{14.12.17 - 25.01.18}%%% Range of date (Planned)
	{Selection of new Cyber Glove, sewing, cabling, soldering.}%%% Taskdescription
	{None.}%%% Issues, solution approach
	{New glove for CyberGlove}%%% Result
		
\task{3D printing}%%% Name of the task
	{hardware}%%% keyword
	{Tanja}%%% People
	{1 hour}%%% Planned time
	{1 hour}%%% Taken time
	{01.02.18}%%% Range of date (Planned)
	{3D Printing of the housing and tracking system privately; the first case (printed at IMI) did not fit and while printing the second case the printer broke, so we had to find a possibility to print the case somewhere else.}%%% Taskdescription
	{None.}%%% Issues, solution approach
	{Successfully printed and suitable housing and tracking system.}%%% Result


%software:	
\subsubsection{Software tasks}
\task{Virtual machine setup}%%% Name of the task
	{software}%%% keyword
	{Johanna, Lukas}%%% People
	{5 hours each}%%% Planned time
	{5 hours each}%%% Taken time
	{09.11.17 - 07.12.17}%%% Range of date (Planned)
	{We wanted set up a virtual machine with PolyVR, VRPN and git installed to be able to work from home and to simulate the Glove. The virtual machine could also be used by other VR Project groups in following semesters.}%%% Taskdescription
	{The building process of PolyVR took much longer than expected in the VM environment. At first the simulation of the Cyberglove did not exactly work out the way we wanted it to, but then we discovered the tool \texttt{netcat} by which the data sent by the Cyberglove can be simulated easily.}%%% Issues, solution approach
	{The result is a working VM which makes working more flexible and more convenient.}%%% Result

\task{Revise WLAN transmission protocol}%%% Name of the task
	{software}%%% keyword
	{Johanna, Lukas}%%% People
	{2 hours each}%%% Planned time
	{1 hour each}%%% Taken time
	{09.11.17 - 16.11.17}%%% Range of date (Planned)
	{Our task was to decide whether we want the Cyberglove to communicate via ZigBee or Wifi (VRPN Protocol). }%%% Taskdescription
	{None.}%%% Issues, solution approach
	{We decided to use Wifi. The reasons for that are that the team from last semester already used Wifi so we knew it would work. Furthermore Wifi offers the opportunity to deliver more data in a single data package than ZigBee. With ZigBee we were not sure whether we could transfer enough data (4 bend sensors, 4 buttons and 2 channels for the joystick), so we  chose Wifi. }%%% Result
	
\task{Implement VRPN library on Arduino}%%% Name of the task
	{software}%%% keyword
	{Johanna, Lukas}%%% People
	{10 hours each}%%% Planned time
	{5 hours each}%%% Taken time
	{09.11.17 - 16.11.17}%%% Range of date (Planned)
	{Our task was to decide whether it is possible to implement the VRPN library completely on the Arduino and if possible, to do so. The current solution was to run a server application on the Cave computer to transfer data between the Cyberglove and PolyVR. In order to evaluate this problem correctly we had to become familiar with the VRPN protocol.}%%% Taskdescription
	{Since the Cyberglove team from last semester already wrote down in their documentation that the Arduino is not compatible with VRPN since the documentation of VRPN is not that good and due to the fact that it would have taken a massive effort to rewrite the VRPN library for our chip we decided to use the server application instead, as did the team in the previous semester. So we decided not to implement the VRPN library. Instead we inserted a command in the PolyVR applications that use the Cyberglove so that the server is started automatically every time the application is started. Something future teams could fix is to shorten the time it takes for the Cyberglove to build up a connection to the server (this currently takes about a minute).}%%% Issues, solution approach
	{Our result is a more or less elegant workaround for the problem.}%%% Result
	
	
\task{Implement software for new hardware}%%% Name of the task
	{software}%%% keyword
	{Johanna, Lukas} %todo more people? %%% People
	{10 hours each}%%% Planned time
	{5 hours each}%%% Taken time
	{14.12.17 - 25.01.18}%%% Range of date (Planned)
	{Since we were going to have a new circuit board the software had to be adapted. For example, pin numbers changed and some bending sensors will be added. In order to make use of those our task was to adapt the hardware software.}%%% Taskdescription
	{None.}%%% Issues, solution approach
	{A working board that is able to address all the new bending sensors and to pass on those values to other applications.}%%% Result
	
	
%usecases:
\subsubsection{Use case tasks}

\task{Getting used to PolyVR}
	{usecase}%%%keyword
	{Ran, Shant} %%%People
	{Ran: 30 hours \\
	 Shant: 5 hours } %%%Planned time
	{Ran: 35 hours \\
	 Shant: 5 hours} %%%Taken time
	{09.11.18 - 23.11.2018} %%% Range of Date
	{Learning how to use PolyVR.}%%%Taskdescription
	{When building the basic environment and adding physic elements, I did not know how to realize it in PolyVR. Then I tried to ask Viktor and Polina for help (Ran). \\ 
	The Marsrover example was not so helpful in my case to get familiar with the PolyVR. Afterwards things were more clarified through  the examples and through Victor's and Polina's explanations whenever I asked them. (Shant)}%%%Issues, solution approach
	{Basic konwledge of the use case development in PolyVR.}%%%Result

\task{First steps with blender}%%% Name of the task
	{usecase}%%% keyword
	{Tanja, Johanna, Frieder}%%% People
	{Tanja: 3 hours, \\
	Johanna: 2 hours,\\
	Frieder: 0 hours}%%% Planned time
	{Tanja: 3 hours,\\
	Johanna: 1 hour,\\
	Frieder: 1 hour}%%% Taken time
	{11.01.18 - 25.01.18}%%% Range of date (Planned)
	{Getting to know blender (with tutorials) and looking for furniture made in blender (Tanja). Adapting hand poses and position (Frieder, Johanna).}%%% Taskdescription
	{None.}%%% Issues, solution approach
	{CAD of furniture and basic knowledge blender (Tanja), appropriate model of virtual hand for Line Drag use case.}%%% Result
	
\task{Line Drag}%%% Name of the task
	{usecase}%%% keyword
	{Johanna, Lukas}%%% People
	{40 hours each}%%% Planned time
	{45 hours each}%%% Taken time
	{02.12.17 - 08.02.18}%%% Range of date (Planned)
	{The goal was to implement a game using all buttons and sensors of the 
	Cyberglove. The idea was to develop an application similar to the game
	``Hot Wire'' where the player has to drag a ball along a given line.}%%% Taskdescription
	{We developed the game using PolyVR. Some issues that occured were that we needed the Cyberglove itself to test the full functionality and the fact that the glove only connects to the PC in the Cave since the PC's IP Adress has been hardcoded in the Arduino Code. But there were no bigger issues, the implementation in general just took us much time.}%%% Issues, solution approach
	{We were able to implement what we wanted to: a fully functional game which uses all the buttons of the Cyberglove. We implemented features like showing penalties and time to complete the level.}%%% Result
	
\task{Model hand poses for Line Drag}
	{usecase}%%%keyword
	{Frieder} %%%People
	{6 hours} %%%Planned time
	{6 hours} %%%Taken time
	{23.01.18} %%% Range of Date
	{Model a 3D hand in the given (by photos) poses, using  a template in blender.}%%%Taskdescription
	{Fingers touching each other was not possible using the controlls  given by the template. Solution: A deeper modification inside the model and meticulous adjustments so that the hand still looks normal were needed. Exporting the model was also exporting "bones" which should not be seen in the final export. Soultion: Find an option to only export the selected objects.}%%%Issues, solution approach
	{8 hands in different poses in the collada format}%%%Result
	
\task{The towers of Hanoi}%%% Name of the task
	{usecase}%%% keyword
	{Fabian, Johanna, Lukas}%%% People
	{Fabian: 20 hours, \\
	 Johanna: 0 hours,\\
	 Lukas: 0 hours}%%% Planned time
	{Fabian 20 hours, \\
	Johanna: 2 hours,\\
	Lukas: 2 hours}%%% Taken time
	{09.11.17 - 08.02.18}%%% Range of date (Planned)
	{Getting into working with Blender and PolyVR and create the towers of hanoi usecase (Fabian). Lukas and Johanna only contributed with some issues with the glove.}%%% Taskdescription
	{Working with PolyVR's physics engine, importing models from Blender.}%%% Issues, solution approach
	{The towers of hanoi usecase.}%%% Result
	
\task{Ball play}
	{usecase}%%%keyword
	{Ran, Johanna, Lukas} %%%People
	{Ran: 40 hours \\
	 Johanna: 0 hours \\
	 Lukas: 0 hours} %%%Planned time
	{Ran: 35 hours \\
	 Johanna: 0.5 hours \\
	 Lukas: 0.5 hours} %%%Taken time
	{01.01.2018 - 01.02.2018} %%% Range of Date
	{Getting into working with Blender and PolyVR and create the ball in sliding usecase. (Ran) Johanna and Lukas just helped with some issues integrating the Cyberglove.}%%%Taskdescription
	{When I finished the 3D Models, they could not be imported correctly in PolyVR (crashed all the time). After adding the UVMap to the models in Blender, it worked. Also I wanted to enable the user to throw the ball, but there is no existing physics acceleration model in PolyVR. It is too hard for me to create one. So I made a sliding board to realize the movement for ball. (Ran)}%%%Issues, solution approach
	{The ball play usecase.}%%%Result

\task{Glove Menu}
	{usecase}%%%keyword
	{Shant} %%%People
	{20 hours} %%%Planned time
	{50 hours} %%%Taken time
	{23.11.17 - 05.02.18} %%% Range of Date
	{Building a use-case for surfing on the web in VR.}%%%Taskdescription
	{To build any use-cases it was necessary to get familiar with PolyVR, Blender, Python, HTML and Javascript. I have learned some Python, HTML and Javascript with online tutorials for beginners. I went through online examples to learn how to use Blender for making 3D objects with Texture and exporting them as Collada, in a way that can be used in PolyVR. The unsolved problem in this use-case was using a virtual keyboard to write in the opened websites. This was not successful. A new virtual keyboard was built that has a textarea in its own window, and text is written there. The text could be copied by the help of a button on the virtual keyboard but could not be pasted in another window without the help of a physical keyboard.}%%%Issues, solution approach
	{This use-case is ready and surfing through websites is working well.}%%%Result
	
\task{Wire and Loop}
	{usecase}%%%keyword
	{Shant} %%%People
	{10 hours} %%%Planned time
	{20 hours} %%%Taken time
	{23.11.17 - 05.02.18} %%% Range of Date
	{Building a 3D game.}%%%Taskdescription
	{To build this use-cases it was necessary to get familiar with game physics. I have read about game physics, collision detection, rigid and solid bodies, degrees of freedom etc. Most of the physics engines online have their software and they use functions that work on their sofware, which can not be applied in PolyVR directly. PolyVR has its own functions that one has to get familiar with by asking and trying. This use-case did not reach its final goal because we could not find a way to make it work as was hoped. Could not move a 3D object (ring) along  a path (curved tube) with simultaneous rotation around its own center.}%%%Issues, solution approach
	{This use-case was left out.}%%%Result

\task{Breakout}
	{usecase}%%%keyword
	{Shant} %%%People
	{10 hours} %%%Planned time
	{10 hours} %%%Taken time
	{23.11.17 - 05.02.18} %%% Range of Date
	{Building a javascript game.}%%%Taskdescription
	{Javascript runs in PolyVR quite smoothly. Some minimal errors appeared and were fixed.}%%%Issues, solution approach
	{This use-case works well.}%%%Result
	
\task{Maze}
	{usecase}%%%keyword
	{Shant} %%%People
	{10 hours} %%%Planned time
	{15 hours} %%%Taken time
	{23.11.17 - 05.02.18} %%% Range of Date
	{To create another 3D game, where a board could be tilted to make a ball move within a maze and be directed toward an exit.}%%%Taskdescription
	{To make the tilting possible a cone was situated below the board and was programmed so that its pointed head functions s a joint, which allows the board to tilt freely. This joint did not function as planned and the game could not be played, despite many tries and follow-ups.}%%%Issues, solution approach
	{This use-case was left aside.}%%%Result

\task{Fruit versus Fastfood}
	{usecase}%%%keyword
	{Shant} %%%People
	{20 hours} %%%Planned time
	{25 hours} %%%Taken time
	{23.11.17 - 05.02.18} %%% Range of Date
	{Building a javascript use-case that exhibits the functionality of the Cyberglove in grabbing things and moving them around. The game was  created according to a tutorial from a book. A sling should be grabbed and pulled and directed toward the target and then released.}%%%Taskdescription
	{The issue here was that the game ran on Mozilla browser, but not in the browser within the PolyVR. The problem was solved by changing the expression "let" with "var" within the Javascript, and by disabling the audio lines within the code.}%%%Issues, solution approach
	{We have a functional game.}%%%Result

%allgemeines zeug
\subsubsection{General effort}
\task{Group Meetings}%%% Name of the task
	{general}%%% keyword
	{All members}%%% People
	{13 hours per person}%%% Planned time
	{19 hours per person}%%% Taken time
	{09.11.17 - 08.02.18}%%% Range of date (Planned)
	{Weekly group meeting to discuss questions and results of each sub group.}%%% Taskdescription
	{None.}%%% Issues, solution approach
	{The group meetings helped us improve the organization and achieve teamwork and communication skills.}%%% Result
	
\task{Lectures}%%% Name of the task
	{general}%%% keyword
	{All members}%%% People
	{11 hours per person}%%% Planned time
	{19 hours per person}%%% Taken time
	{19.10.17 - 08.02.18}%%% Range of date (Planned)
	{Organizational information, demonstrations in cave and of all VR projects, presentations of homework, group project plan, mid term presentations and final project presentation.}%%% Taskdescription
	{None.}%%% Issues, solution approach
	{Teamwork skills, communication, overview of current project state.}%%% Result
	
\task{PolyVR Lab / Tutorial}%%% Name of the task
	{general}%%% keyword
	{All team members}%%% People
	{3 hours per person}%%% Planned time
	{3 hours per person}%%% Taken time
	{19.10.17 - 02.11.17}%%% Range of date (Planned)
	{Introduction in PolyVR with exercises.}%%% Taskdescription
	{None.}%%% Issues, solution approach
	{Basic skills in PolyVR.}%%% Result
	
\task{Seminar with Michael and preparation of presentation and Gantt chart}%%% Name of the task
	{general}%%% keyword
	{All team members}%%% People
	{3 hours per person}%%% Planned time
	{3 hours per person}%%% Taken time
	{19.10.17 - 02.11.17}%%% Range of date (Planned)
	{Understanding the project tasks.}%%% Taskdescription
	{None.}%%% Issues, solution approach
	{Clear task definition for the project, method and working steps, decision for WLAN (VRPN) as transmission protocol, structure project plan presentation, Gantt chart. }%%% Result
	
\task{Preparation of homework presentation}%%% Name of the task
	{general}%%% keyword
	{All team members}%%% People
	{Tanja: 4 hours, \\
	Johanna: 5 hours,\\
	Lukas: 5 hours,\\
	Fabian: 3 hours,\\
	Frieder: 3 hours,\\
	Ran: 4 hours,\\
	Shant: 15 hours}%%% Planned time
	{Tanja: 8 hours,\\ 
	Johanna: 8 hours,\\
	Lukas: 8 hours,\\
	Fabian: 2 hours,\\
	Frieder: 8 hours,\\
	Ran: 5 hours,\\
	Shant: 15 hours}%%% Taken time
	{19.10.17 - 26.10.17}%%% Range of date (Planned)
	{Each team member prepared his or her presentation about the individually assigned topic. }%%% Taskdescription
	{None.}%%% Issues, solution approach
	{Presentation about the respective topic.}%%% Result


\task{Preparation of 1st presentation}%%% Name of the task
	{general}%%% keyword
	{All team members}%%% People
	{Tanja: 5 hours, \\
	Johanna: 3 hours,\\
	Lukas: 2 hours,\\
	Fabian:  1 hour,\\
	Frieder: 2 hours,\\
	Ran: 1 hour,\\
	Shant: 1 hour}%%% Planned time
	{Tanja: 5 hours, \\
	Johanna: 1 hour,\\
	Lukas: 2 hours,\\
	Fabian:  1 hour,\\
	Frieder: 2 hours,\\
	Ran: 1 hour,\\
	Shant: 1 hour}%%% Taken time
	{02.11.17 - 09.11.17}%%% Range of date (Planned)
	{Development the objectives for the project term, dividing the objectives in sub groups and scheduling the task as Gantt chart.}%%% Taskdescription
	{None.}%%% Issues, solution approach
	{Presentation.}%%% Result
	
\task{Preparation of midterm presentation}%%% Name of the task
	{general}%%% keyword
	{All team members.}%%% People
	{Tanja: 4 hours, \\
	Johanna: 3 hours, \\
	Lukas: 2 hours, \\
	Fabian:  2 hours, \\
	Frieder: 2 hours,\\
	Ran: 10 hours,\\
	Shant: 2 hours}%%% Planned time
	{Tanja: 4 hours,\\ 
	Johanna: 2 hours,\\
	Lukas: 4 hours,\\
	Fabian: 2 hours,\\
	Frieder: 2 hours,\\
	Ran: 12 hours,\\
	Shant: 7 hours}%%% Taken time
	{07.12.17 - 14.12.17}%%% Range of date (Planned)
	{Preparation the mid term presentation with a summary of fulfilled tasks refering to time schedule (Gantt chart) and giving an outlook of the next steps and tasks.}%%% Taskdescription
	{None.}%%% Issues, solution approach
	{Presentation.}%%% Result
	
\task{Preparation of final presentation}%%% Name of the task
	{general}%%% keyword
	{All team members.}%%% People
	{Tanja: 3 hours, \\
	Johanna: 2 hours,\\
	Lukas: 2 hours,\\
	Fabian: 3 hours,\\
	Frieder: 2 hours,\\
	Ran: 5 hours,\\
	Shant: 15 hours}%%% Planned time
	{Tanja: 3 hours,\\ 
	Johanna: 3 hours,\\
	Lukas: 2 hours,\\
	Fabian: 3 hours,\\
	Frieder: 2 hours,\\
	Ran: 5 hours,\\
	Shant: 15 hours}%%% Taken time
	{01.02.18 - 08.02.18}%%% Range of date (Planned)
	{Preparation of final presentation, accomplished goals, outlook of future development possibilities of the CyberGlove. Preparation of Use Case presentation in the Cave.}%%% Taskdescription
	{None.}%%% Issues, solution approach
	{Presentation.}%%% Result

\task{Documentation of our work}%%% Name of the task
	{general}%%% keyword
	{All team members.}%%% People
	{Tanja: 8 hours, \\
	Johanna: 10 hours,\\
	Lukas: 10 hours,\\
	Fabian: 10 hours,\\
	Frieder: 5 hours,\\
	Ran: 9 hours,\\
	Shant: 5 hours}%%% Planned time
	{Tanja: 8 hours,\\ 
	Johanna: 15 hours,\\
	Lukas: 15 hours,\\
	Fabian: 8 hours,\\
	Frieder: 6 hours,\\
	Ran: 7 hours,\\
	Shant: 5 hours}%%% Taken time
	{25.01.18 - 08.02.18}%%% Range of date (Planned)
	{Documentation of all of our tasks that we worked on during the semester. We also described our learning outcomes, issues and project results.}%%% Taskdescription
	{}%%% Issues, solution approach
	{Full documentation to hand in.}%%% Result

%organisation
\subsubsection{Coordination and management}
\task{Project Management and Coordination}%%% Name of the task
	{coordination}%%% keyword
	{Tanja}%%% People
	{5 hours}%%% Planned time
	{5 hours}%%% Taken time
	{09.11.17 - 08.02.18}%%% Range of date (Planned)
	{Communication, organization, coordination.}%%% Taskdescription
	{None.}%%% Issues, solution approach
	{Fulfilling of tasks in time, good communication.}%%% Result


\task{Gantt chart}%%% Name of the task
	{coordination}%%% keyword
	{Tanja}%%% People
	{4 hours}%%% Planned time
	{6 hours}%%% Taken time
	{02.11.17 - 08.02.18}%%% Range of date (Planned)
	{Creating and updating Gantt chart.}%%% Taskdescription
	{None.}%%% Issues, solution approach
	{Updated Gantt chart.}%%% Result
	
\task{Buying new glove}%%% Name of the task
	{coordination}%%% keyword
	{Tanja}%%% People
	{1 hour}%%% Planned time
	{1.5 hours}%%% Taken time
	{25.01.18}%%% Range of date (Planned)
	{Buying new glove.}%%% Taskdescription
	{None.}%%% Issues, solution approach
	{New glove.}%%% Result

\task{Preparation of project video}%%% Name of the task
	{coordination}%%% keyword
	{Tanja}%%% People
	{15 hours}%%% Planned time
	{25 hours}%%% Taken time
	{14.12.17 - 08.02.18}%%% Range of date (Planned)
	{Getting used to work with Adobe Prime, video shoot, editing.}%%% Taskdescription
	{None.}%%% Issues, solution approach
	{Finished Project video.}%%% Result
	
\task{Creating poster and flyer}%%% Name of the task
	{coordination}%%% keyword
	{Tanja}%%% People
	{6 hours}%%% Planned time
	{8 hours}%%% Taken time
	{14.12.17 - 08.02.18}%%% Range of date (Planned)
	{Creating flyer and poster with pictures and information about the Cyberglove.}%%% Taskdescription
	{None.}%%% Issues, solution approach
	{Flyer and poster of CyberGlove project.}%%% Result
	
	
%%% Name of the task
%%% keyword
%%% People
%%% Planned time
%%% Taken time
%%% Range of date (Planned)
%%% Taskdescription
%%% Issues, solution approach
%%% Result