\section{Project coordination and Management}

The following chapter deals with the tasks of project coordination and management and explains the difficulties, problems as well as their solution, which influenced the work flow through the Cyber Glove project.

\subsection{Tasks, problems, and difficulties}

First of all it was necessary to find a way to combine the knowledge and skills of students from different disciplines, to create openness to learn together and from each other, and to transform the shared resources into an interesting and goal fulfilling project outcome.

Forming a group out of students from different disciplines but also from different countries with different culture and language backgrounds made communication and management within this interdisciplinary group framework sometimes even more difficult. Learning from each other and approaching each other when problems arise is not a prerequisite in every culture. After realizing those difficulties the focus was set on improving communication and understanding within the group. 

Individual motivation, increased communication effort as well as control was necessary and helpful. Furthermore, the approach to the given tasks was different in many aspects due to the different disciplines, characters and cultures. Again it was important to first analyze the circumstances and then make decisions to merge.

In addition it was necessary to frequently communicate the current project state, problems as well as uncertainties with the supervisor. The fact that our supervisor was often not available in person and communication by mail took longer than calculated made fulfilling the tasks long-winded and difficult. After realizing, communication and meetings were scheduled with lead time and we tried to fulfill tasks with the resources we could organize ourselves. 

Apart of facing those tasks and difficulties, weekly meetings were organized and held, fulfillment of tasks was discussed and new tasks as well as goals were defined. While keeping an eye on the project schedule it was necessary to focus on an equal and fair division of work within the group. 

In addition to the tasks named above 3D printing as well as designing the basic parts of the new Cyber Glove among other things by sewing, soldering and cabling were parts of the Project Management.


\section{Summary and outlook}
In the previous semester a Cyber Glove was developed to enable the integration of intuitive finger and hand gestures into a virtual reality scene. The bending and the contact of the fingers as well as the hand position in the scene were tracked. 

In the Virtual Reality Practical Course in the winter semester 2017/2018 our team achieved designing and developing a new Cyber Glove to improve the functionality and usage of it and to make it more comfortable to wear. Hereby we designed an advanced version of the board, housing, wiring and tracking system as well as implemented three additional bending sensors and finger sensors. Furthermore we created an one plug solution and adapted the Aduino code to the new hardware.

The new connection system of the housing with the glove, based on magnets, enables disconnecting one from the other. This simplifies working on the housing, board and wiring or exchanging components during future development of the Cyber Glove.

Another innovation were the new Use Cases our group created and programmed with PolyVR. Now users can chose in-between the cases ‘Line drag’, ‘Tower of Hanoi’, ‘Ball playing’, ‘Menu’ and others.

Looking at the future development opportunities of the Cyber Glove one recommendation is adding an attachment for the joystick which would improve the handling. Another one is shortening the cables. In addition we suggest shortening the time the Glove needs to connect with the server application to exchange data between the Glove and PolyVR. 
Moreover charging the battery via the microcontroller is recommended to be fixed.

Further development in the area of use cases could be realized by creating new use cases which are more complex. For example referring to the use case ‘line drag’: more levels of increasing difficulty could be created and a database to save the user scores to compare them with each other could be connected.
